\section{Simulador de \emph{malware}}
\label{sec:sim}

Nesta seção apresentamos o simulador construído para validar
empiricamente a precisão do nosso modelo analítico.  Nosso simulador
é flexível e permite a avaliação de \emph{malware} com diferentes
padrões de comportamento.

\subsection{Configuração da rede e dispositivos}

Nosso simulador permite a configuração da rede e dos dispositivos
dos parâmetros mostrados na tabela~\ref{tab:sim.params}.  Em
particular, o simulador permite execução de redes com diferentes
quantidades $N_v$ e $N_s$ de dispositivos vulneráveis
e invulneráveis, i.e., que podem e não podem ser infectados pelo
\emph{malware}.  Também é possível configurar a quantidade $N_m$ de
dispositivos que executam o \emph{malware} continuamente.  Os
dispositivos vulneráveis e invulneráveis poder ser desligados
e religados.  A duração dos períodos nos quais os dispositivos
permanecem ligados e desligados é escolhida aleatoriamente segundo
distribuições estatísticas, $D_\textrm{on}$ e $D_\textrm{off}$.
Para capturar a volatilidade de \emph{malwares} comuns em
dispositivos IoT, que não possuem armazenamento persistente,
o simulador considera que um dispositivo vulnerável volta
à configuração de fábrica (i.e., não infectado) quando é religado.

\begin{table}
	\begin{center}
		\begin{tabular}{ll}
			\textsc{Param.} & \textsc{Descrição} \\
			\hline
			\multicolumn{2}{l}{\emph{Tamanho da rede}} \\
			$N_s$ & Número de dispositivos não vulneráveis \\
			$N_v$ & Número de dispositivos vulneráveis \\
			$N_m$ & Número de dispositivos executando o malware \\
			\hline
			\multicolumn{2}{l}{\emph{Comportamento dos dispositivos}} \\
			$D_\textrm{on}$ & Distribuição do período ligado (on-time) \\
			$D_\textrm{off}$ & Distribuição do período desligados (off-time) \\
			\hline
			\multicolumn{2}{l}{\emph{Latência fim-a-fim}} \\
			$l_\textrm{min}$ & Latência fim-a-fim mínima \\
			$l_\textrm{max}$ & Latência fim-a-fim máxima \\
			$l_\textrm{tout}$ & \emph{Timeout} de conexão \\
			$R_\textrm{auth}$ & Distribuição de RTTs em uma tentativa de autenticação \\
			$R_\textrm{infect}$ & Distribuição de RTTs em uma tentativa de infecção \\
		\end{tabular}
		\caption{Parâmetros de configuração da rede e dos dispositivos no simulador.}
		\label{tab:sim.params}
	\end{center}
\end{table}

O simulador calcula uma latência de comunicação fim-a-fim distinta
para cada par de dispositivos na rede. A latência fim-a-fim para
cada par de dispositivos é uniformemente distribuída entre
$[l_\textrm{min}, l_\textrm{max}]$.  A latência fim-a-fim entre um
dispositivo executando o \emph{malware} e um dispositivo alvo
é utilizada para calcular o tempo total de uma tentativa de
autenticação e de uma tentativa de infecção.  Em particular,
multiplicamos a latência fim-a-fim pela quantidade de trocas de
mensagens fim-a-fim (RTTs) em tentativas de autenticação e em
tentativas de infecção.  A quantidade de mensagens em tentativas de
autenticação e infecção é segundo distribuições estatísticas
($R_\textrm{auth}$ e $R_\textrm{infect}$).  Por último,
$l_\textrm{tout}$ controla o intervalo no qual o \emph{malware}
espera por uma resposta do dispositivo alvo antes de desistir da
tentativa de autenticação ou de infecção.  Em particular,
dispositivos invulneráveis ou previamente infectados não respondem
a tentativas de autenticação e de infecção.

\subsection{Configuração e comportamento do \emph{malware})

O simulador implementa diferentes variações de comportamento do
processo de sondagem e do mecanismo de escolha de alvos em
\emph{malwares}.  O simulador considera que todos os dispositivos
infectados executam a mesma configuração do \emph{malware} (e.g.,
participam da mesma \emph{botnet}).

O processo de sondagem controla como o \emph{malware} gera novas
tentativas de autenticação e infecção.  O simulador implementa três
processos de sondagem:

\begin{description}
%
	\item[FixedRate] No processo FixedRate, o \emph{malware}
		cria tentativas de autenticação e infecção a uma
		taxa fixa, independente da duração ou sucesso das
		tentativas.
%
	\item[FixedThreads] No processo FixedThreads,
		o \emph{malware} executa uma quantidade fixa de
		\emph{threads} que executam uma tentativa de
		autenticação e, em caso de sucesso, uma tentativa de
		infecção.  Após as tentativas de autenticação
		e infecção, a \emph{thread} escolhe um novo alvo
		e repete o processo.
%
	\item[FixedForkingThreads] No processo FixedForkingThreads,
		o \emph{malware} executa uma quantidade fixa de
		\emph{threads} que executam uma tentativa de
		autenticação e, em caso de sucesso, disparam uma
		nova \emph{thread}$^\prime$ independente para
		executar a tentativa de infecção.  Após a tentativa
		de autenticação, a \emph{thread} escolhe novo alvo
		e repete o processo.
%
\end{description}

Tentativas de autenticação podem falhar por que o dispositivo alvo
é invulnerável ou por que já foi infectado.  Em ambos os caso,
o dispositivo alvo não responde à tentativa de autenticação, que
falha depois de $l_\mathrm{tout}$ segundos.  Tentativas de infecção
podem falhar por que o dispositivo vulnerável foi infectado por
outro \emph{malware} executando em paralelo.  (Quando múltiplos
\emph{malwares} tentam infectar um dispositivo vulnerável ao mesmo
tempo, apenas um irá obter sucesso.)

O mecanismo de escolha de alvos controla como o \emph{malware} gera
novos alvos para infecção.  O simulador implementa três mecanismos
para geração de alvos:

\begin{description}
%
	\item[NoCache] No mecanismo NoCache, alvos são selecionados
		aleatoriamente entre todos os dispositivos.
%
	\item[LocalCache] No mecanismo LocalCache, o \emph{malware}
		mantém uma lista de dispositivos sondados
		anteriormente onde tentativas de autenticação ou
		infecção falharam.  Dispositivos são removidos da
		lista após um período configurável para permitir
		reinfecção de dispositivos religados.  Alvos que não
		estão na lista são selecionados aleatoriamente.
%
	\item[GlobalCache] Similar ao mecanismo LocalCache, mas os
		\emph{malwares} compartilham informação sobre quais
		dispositivos sondaram.
%
\end{description}

